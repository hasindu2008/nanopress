\section{The Focus}

%TODO why lossless important
%TODO why not lossy
%TODO why per read

It is important for fast parallel querying that the reads are compressed separately. That's not to say that each read is independent. Reads originating from the same channel and well/pore will have been recorded by the same sensor and so will likely have similar properties. This is especially true for reads recorded at a similar timestamp as each channel deteriorates over the course of a sequencing run. A compression method may take advantage of these available patterns but it is important for querying that each read can be decompressed independently. Perhaps there exists a multi-read compression strategy which is great for space reduction and hence archival purposes. However, such a strategy is not obvious and the primary desire of bioinformaticians is a better compression algorithm that does not sacrifice parallel querying. Hence, multi-read compression is outside the scope of this thesis.

Lossy compression, although an interesting avenue, is not desirable for most research applications where accuracy is more valued than storage space. For example, long-term archival of scholarly research data must be lossless to allow for public replication of any results. For this reason, lossless compression is the focus of this thesis.
