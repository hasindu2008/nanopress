\subsection{Range Coding}

Now consider encoding the one byte data using range coding rather than Huffman
coding. Recall that range coding encodes all of the data into one number. To
begin with, an initial range of numbers is chosen. Then, for each symbol in the
data stream the current range is narrowed down based on the current symbol's
probability of occurrence. A representative number from the final range is then
chosen as the output.

The probability distribution of the data can be predetermined, calculated by an
initial pass or predicted adaptively. Most available implementations use models
for predicting the probability distribution of the next symbol.
Some common models include order-0, order-1 and order-2 models, context mixing and
secondary symbol estimation (SSE). These models are used during encoding and
decoding meaning that no distribution needs to be encoded with the compressed
data.

The order-$n$ model gives the probability distribution of the next symbol given
the previous $n$ symbols (known as the \textit{context}). It is updated at each
step to increase the revealed symbol's probability when the same context appears
again.
%For example, by the end of the algorithm the order-0 model would have calculated
%the probability distribution of the one byte data, whilst the order-1 model
%would have calculated a conditional probability distribution for each value in
%the data.
Basically the order-$n$ model queries and updates a frequency table of
size $S^{n+1}$ where $S$ is the size of the symbol's alphabet.
It finds the probability of the next symbol by querying the table and
dividing this count by the total count of the symbols read thus far. To avoid a
probability of 0 occuring, every value in the table is initialised to the
same count -- typically a small value between 2 and 10 or a much larger
value such as $2^{14}$ as in TurboRC.
There are bytewise and bitwise models which operate at the byte and bit level
respectively. Bytewise models operate over symbols from the alphabet
$\{0,1,\dots,255\}$ whilst bitwise models keep track of the frequency of
the binary digits 0 and 1. Let the range coding algorithms which use the order-0
and order-1 models respectively be named \textit{rc0} and \textit{rc1}.

Context mixing algorithms combine two models to yield higher accuracy
predictions. They predict one bit at a time and are usually based on one of
two mixing algorithms: linear or logistic mixing. Linear mixing performs
a weighted average of the models whilst logistic mixing transforms the
probabilities using logistic modelling which gives more weight to
probabilites closer to 0 and 1.

Secondary symbol estimation (SSE) is a prediction postprocessing approach which
uses the combined prediction of context mixing and a small context to generate a
refined prediction.

Consider the range coding algorithm which applies order-0 and order-1
context mixing with SSE. We'll abbreviate this strategy to \textit{rc01s}. The
TurboRC implementation initialises the order-0, order-1 and SSE models in $O(1)$
time. Then, for each byte it iterates over its bits and updates the range
followed by the models. A normalisation procedure built into the encoding
algorithm is required to ensure that there is always a large enough range for
further sub-ranges. Overall, the algorithm is $O(n)$ but since it iterates over
bits rather than bytes and there are many update procedures -- it is slower than
huff in practice. The same is true for decoding. However, the compression ratio
of rc01s is estimated to be better than huff and shuff. This is because range
coding is as or more optimal as Huffman coding but without the drawback that
each code length must be a whole number of bits. Furthermore, combined with
accurate modelling range coding performs just as well without requiring the
storage of a probability distribution as in huff.

% http://mattmahoney.net/dc/dce.html
