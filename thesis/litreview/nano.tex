The primary interest of bioinformatics is to understand biological data. This is typically achieved through the development of biology-aware software tools. Nanopore sequencing, which emerged in the 1980s, is one approach employed in bioinformatics to determining the primary structure of biopolymers such as DNA and RNA. It involves recording the ionic current as a biopolymer passes through a protein pore the scale of a nanometer with a voltage applied to its surrounding membrane. See Figure \ref{fig:nano} for a pictorial overview of nanopore sequencing. The resulting ionic current signal is then used to determine the nucleotide sequence of the biopolymer. Oxford Nanopore Technologies (ONT) is the leading producer of nanopore sequencing machines and is used in most peer-reviewed studies involving nanopore sequencing. Several ONT sequencing machines are commercially available including (in order of throughput) the MinION, GridION and PromethION.
Each machine uses one or many flow cells to sequence DNA or RNA molecules. Each \textit{flow cell} contains thousands of nanopores which are grouped into fours with each group corresponding to its own channel. Each channel is assigned an electrode in the flow cell's sensor chip which is controlled and measured by an application-specific integrated circuit (ASIC) chip.
The signal recorded by such machines is digitised and recorded as a sequence of 16-bit signed integers. This sequence, hereafter referred to as \textit{nanopore signal data}, is the focus of this thesis.

The primary interest of bioinformatics is to understand biological data. This is typically achieved through the development of software tools. Nanopore sequencing is one approach employed in bioinformatics to determining the primary structure of biopolymers such as DNA and RNA. It involves recording the ionic current as a biopolymer passes through a pore of nanometer size with a voltage applied to the membrane. The resulting current signal is then used to determine the structure of the biopolymer. Oxford Nanopore Technologies (ONT) is the leading producer of nanopore sequencing machines. The signal recorded by such machines is digitised and given as a sequence of 16-bit signed integers. This sequence, hereafter referred to as \textit{nanopore signal data}, is the focus of this thesis.

There have been many studies investigating genomic data compression \cite{genomic-comp}. Few, however, have successfully focussed on losslessly compressing the signal data produced directly from nanopore sequencing. One tool called Picopore does not produce any novel insights, but rather, increases the level of gzip compression to the highest possible \cite{picopore}. VBZ, described in section \ref{sec:data-other}, is the current state-of-the-art encoding introduced by ONT in 2019.

Rather interestingly, there has been some recent research effects to investigate lossy compression of nanopore signal data. Chandak et. al. found that lossy time-series compressors (LFZip and SZ) tend to have a very small impact on the downstream analysis of nanopore data \cite{lossy-nano, lfzip}. In particular, after reducing the input size by 35-50\%, basecalling and consensus accuracy is reduced by less than 0.2\% and 0.002\% respectively. However, they seem to overlook the prospect of a more efficient lossless compression technique for nanopore signal data:
\begin{displayquote}
``obtaining further improvements in lossless compression (to VBZ) is challenging due to the inherently noisy nature of the current measurements'' \cite{lossy-nano}.
\end{displayquote}
For this reason and perhaps due to the novel nature of nanopore sequencing, little has been further attempted in the literature to losslessly compressing nanopore signal data.

Fortunately, signal data similar in form to nanopore signal data is commonly recorded and extensively researched. Some examples include electrograms of the brain (EEG) and heart (ECG); seismic and radio waves; telemetry data from astronomy; and sonar signals. MCDRC is a deep learning approach to losslessly compressing sensor signal data based on a recurrent neural network architecture known as a multi-channel recurrent unit \cite{mcdrc}. Its results are quite promising, outperforming existing techniques such as BSC, PAQ and CMIX.

% TODO put footnote in right place
\footnotetext[3]{Figure \ref{fig:nano} was taken from Figure 1 in \cite{Wang2021}.}
