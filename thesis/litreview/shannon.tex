Shannon's ``A Mathematical Theory of Communication'' \cite{shannon} is the
founding article for the field of information theory. It investigates the
theoretical properties underpinning discrete and continuous communication
systems with and without introduced noise.

The nanopore sequencing pipeline can be decomposed to align with Shannon's
general communication system. See Figure~\ref{fig:shannon-comm-sys}
The \textit{information source} is the DNA/RNA sequence of a sample.
The \textit{transmitter} is the nanopore sequencing machine which produces the
signal. In our case, we desire to further encode this signal using fewer bits.
This is known as data compression.
The \textit{channel} is %TODO
The \textit{receiver} is the software basecalling pipeline used to reconstruct
the sequence.
The \textit{destination} is a scientist or customer.

%TODO
\begin{figure}
\caption{The nanopore sequencing pipeline.}
\end{figure}

The nanopore sequencing pipeline is comparable to Shannon's discrete noiseless
system.

A nanopore sequencer can be thought of as a stochastic process. It produces a
discrete sequence of integers governed by a set of probabilities.
