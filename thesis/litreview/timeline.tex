A timeline of the major events in the data compression and nanopore sequencing literature is presented in Table \ref{tab:lit}.

\begin{table}
\centering
\caption{A timeline of the major events in the literature.}
\label{tab:lit}
\begin{tabular}{r l}
\hline
1948 & Shannon publishes ``A Mathematical Theory of Communication'' \cite{shannon}\\
1952 & Huffman publishes a method for finding optimal prefix codes \cite{huffman}\\
1967 & Run-length encoding first described \cite{rle}\\
1976 & Pasco and Rissanen develop arithmetic coding independently \cite{arithmetic-coding}\\
1977 & Lempel and Ziv publish LZ77 \cite{lz77}\\
1978 & Lempel and Ziv improve upon LZ77 with LZ78 \cite{lz78}\\
1979 & Martin proposes range coding \cite{range-coding}\\
1984 & Welch publishes LZW based on LZ78 \cite{lzw}\\
1987 & Vitter improves upon the adaptive Huffman coding algorithm FGK \cite{vitter}\\
1994 & Burrows and Wheeler publish the Burrows-Wheeler transform \cite{bwt}; and \\
& Gage submits byte-pair encoding as a C article \cite{byte-pair}\\
2014 & ONT release the first portable nanopore sequencing device\\
2017 & A tool for reducing nanopore data, Picopore, is released \cite{picopore}\\
2018 & Lemire publishes Stream VByte \cite{svb}\\
2019 & VBZ is first released\\
2020 & Lossy compression of nanopore signal data is investigated \cite{lossy-nano}\\
2021 & MCDRC is published \cite{mcdrc}\\
\hline
\end{tabular}
\end{table}
