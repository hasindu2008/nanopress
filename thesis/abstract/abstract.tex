\chapter*{Abstract}

Storing DNA digitally is expensive as it heavily consumes disk space. Hence,
compression of this DNA data is highly desirable. However, for efficient
analysis of this data, decompression time must not be overwhelming. A balance
between the compression rate and decompression time should be taken.

Contributions
\begin{itemize}
\item first thorough analysis of nanopore DNA signal data including
	understanding its characteristic sections.
\item first known software benchmark of nanopore signal compression methods.
\item vbbe21: an encoding container for nanopore signal data which improves the
	compression ratio of downstream compression methods.
It encodes the one-byte data as is and the two-byte `exceptions' are bit packed.
It has the same time complexity for both encoding and decoding as the
state-of-the-art svb(16).
\item static huffman: Huffman compression method with a higher compression ratio than
	the state-of-the-art which uses a predetermined table and tree.
\item range coding: a compression method with an even higher compression ratio
	than static huffman using order-0 and order-1 context mixing followed by
	SSE
\item stall encoding: dynamic domain-specific frame of reference encoding of the
	stall which can be paired with another method on the non-stall.
\item new state-of-the-art method which combines stall encoding, vbbe21 and
	range coding.
\end{itemize}
