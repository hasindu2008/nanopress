\chapter*{Abstract}

Lossless compression of nanopore DNA signal data with more space saving than the
state-of-the-art is highly desirable due to the magnitude and rate of research
and clinical nanopore data being produced.
Nanopore DNA sequencing produces a characteristic signal which represents the
ionic current over time as a molecule's DNA strand passes through a nanoscale
protein pore.
This signal is used to determine the order of DNA nucleobases (A,
C, T \& G) in the molecule; useful for medical diagnosis and forestic biology
amongst other applications.
Data compression is necessary for reducing the signal's long-term storage costs
which are estimated to be in the order of an extra \$6000 each year for a medium
clinical research facility.

In this thesis, we provide a systematic analysis of nanopore DNA signal data
including the articulation of its characteristic features and transformations.
The problem space of suitable compression methods is theorised and several novel
approaches to compressing nanopore signal data are detailed including a novel
variable byte encoding vbbe21, a static Huffman encoding, optimal subsequence
searching, stall encoding and separation into the jumps, falls and flats.
The first comprehensive benchmark of existing and novel compression methods for
nanopore signal data is conducted on the NA12878 data set.
A new state-of-the-art space saving of 0.666 is achieved using dstall-fz
compared to the previous best of 0.659. This combines vbbe21 with dynamic stall
encoding and range coding modelled by order-0 order-1 secondary symbol
estimation (SSE).

For archiving purposes, it is advised to use the dstall-fz-1500 method which
closely approximates the decision boundary of dstall-fz in half the compression
time -- achieving a space saving of 0.666. For faster compression and
decompression speed than dstall-fz-1500 but better compression than the previous
state-of-the-art, rc1-vbbe21-zd is recommended which achieves a space saving of
0.661. If analysis speed is the utmost priority, it is advised to continue using
the previous state-of-the-art zstd-svb-zd until faster implementations of the
articulated methods are available.

%Nanopore DNA sequencing is one approach used to determine the order of DNA
%nucleobases (A, C, T \& G) in a molecule

%These data sets are often sensitive

%Storing DNA digitally is expensive as it heavily consumes disk space. Hence,
%compression of this DNA data is highly desirable. However, for efficient
%analysis of this data, decompression time must not be overwhelming. A balance
%between the compression rate and decompression time should be taken.

