\chapter{Conclusion} \label{chap:conclusion}

Nanopore sequencing is a next-generation approach to DNA sequencing known for
its speed, portability, long reads and real-time analysis. Its increasing
popularity has been further influenced by the advent of the COVID-19 pandemic
which resulted in approximately one quarter of all SARS-CoV-2 virus genomes
being sequenced by nanopore devices \cite{lara}. This popularity has caused a
dramatic increase in data storage costs pertaining to the magnitude of the data
which is around 1 TB per human DNA sequencing run.
The current approach to compressing this data does not consider its
unique characteristics and thus misses crucial space saving opportunities.

As a result, we set out to discover lossless compression methods for archiving
nanopore signal data which achieve more space saving that the current
state-of-the-art.
Data compression is fundamentally an artificial intelligence problem of
understanding the data and making the correct predictions. Hence, after
providing the background material for the thesis, we conducted a systematic
analysis of the signal data including an examination of its characteristics and
suitable transformations.
Then, we explored several strategies which exploit the signal data's
characteristics including the vbbe21-zd encoding, static Huffman, optimal
subsequence searching, stall-specific encoding and separating the jumps and
falls from the flats.
After setting up the first comprehensive benchmark of existing and novel
compression methods for nanopore signal data, a large experiment was conducted
wherein each read from the downsampled NA12878 data set was compressed and
decompressed sequentially using each method.
Space and time metrics were collected and a superior space saving of 0.666
versus 0.659 -- which outperforms the entropy of the deltas -- was achieved
using the dstall-fz method. This strategy combines vbbe21 with dynamic stall
encoding and range coding modelled by order-0 order-1 SSE. Furthermore, an
alternative dynamic stall encoder, dstall-fz-1500, was found to approximate the
decision boundary of dstall-fz whilst achieving the same space saving of 0.666 in
half the compression time.
