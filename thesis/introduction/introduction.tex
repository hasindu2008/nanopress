\chapter{Introduction}

% Explain what I am doing and why to my Mum...
% Outline of Chapters
	% One sentence per chapter
% Contributions
% 	Evaluate current state of the art
% 	Testing environment setup (eg github)
% 	Design and implement superior algo

The DNA sequence of an organism encodes the information necessary for its
survival and reproduction.
Because of its key role in living organisms, DNA sequencing has become a
burgeoning field in both academic research and clinical medicine.
%Knowledge of this sequence can be used to identify,
%diagnose and treat genetic diseases.
Nanopore sequencing is one such approach used to determine the order of
nucleobases (A, C, T and G) in a DNA molecule.

Nanopore sequencing records the ionic current as a DNA molecule passes through a
nanoscale protein pore (or \textit{nanopore}). Disturbances in the resulting
current signal are used to determine the nucleobases of the DNA molecule.

The problem is that nanopore sequencing produces a huge volume of data which is
being recorded at an increasing rate and often needs to be archived for several
years due to certain data regulations.
For instance, the Genomic Technologies Group at the Garvan Institute of Medical
Research regularly perform between 5 and 10 clinical human DNA sequencing runs a
week which need to be archived for up to 10 years. This amounts to between 260
and 520 TiB a year or up to an additional USD \$6000 a year when archiving with
Google Cloud Storage. Over 10 years this cost accumulates to roughly USD
\$\num{330000}.

Data compression solves this problem by removing statistical redundancies in the
data and hence decreasing its size.
Lossless data compression achieves
this without losing any information in the process. Decompression can then be
performed to reobtain the original data. Data compression was founded on the
principles of information theory laid out by Shannon in 1948. Entropy encoding
is one such technique which encodes each symbol with a binary code. Huffman
coding is a famous example which encodes the most frequently occuring symbols with the
smallest codes and the least frequent with the largest. Arithmetic or range
coding is another such method which encodes a sequence of symbols as one number
within some range. These methods attempt to approach the entropy of the data
which represents the level of uncertainty inherent in the data's distribution.

The state-of-the-art method used to compress nanopore signal data takes the
differences between each successive point, applies a 

% Past methods

%Hence, compressing the data as much as possible without losing any information
%is highly desired.

%It typically produces around 1 TiB of data for
%a human DNA sequencing run.
% How many runs by some well known institute or project to put into perspective?
% G42 (60 promethions 50 samples a week per prom = 3000 samples), Nottingham, UCSC, Washington
%The Genomic Technologies Group at the Garvan Institute of Medical Research
%regularly perform 5-10 clinical human DNA sequencing runs a week which need to
%be archived for up to 10 years. This amounts to between 260 and 520 TiB a year.
% How much data produced in certain years, for different projects, etc. ?
% normal research 5 years
% medical related 5-10 years
% NHMRC ARC uni data regulations

%Around 3 billion bases in the human genome.
%Each base is recorded by roughly 10 signal points (=30 billion points).
%2 bytes per integer (=60 GiB)
%To increase sequencing accuracy each base is recorded many times over. For 20 x coverage (=1 TiB)
%Many labs doing human DNA sequencing at a large rate.

\section{Contributions}
\begin{itemize}
\item first thorough analysis of nanopore DNA signal data including
	understanding its characteristic sections.
\item first known software benchmark of nanopore signal compression methods.
\item vbbe21: an encoding container for nanopore signal data which improves the
	compression ratio of downstream compression methods.
It encodes the one-byte data as is and the two-byte `exceptions' are bit packed.
It has the same time complexity for both encoding and decoding as the
state-of-the-art svb(16).
\item static huffman: Huffman compression method with a higher compression ratio than
	the state-of-the-art which uses a predetermined table and tree.
\item range coding: a compression method with an even higher compression ratio
	than static huffman using order-0 and order-1 context mixing followed by
	SSE
\item stall encoding: dynamic domain-specific frame of reference encoding of the
	stall which can be paired with another method on the non-stall.
\item new state-of-the-art method which combines stall encoding, vbbe21 and
	range coding.
\end{itemize}

The first thorough analysis of nanopore DNA signal data was conducted including
understanding its characteristic sections. A new state-of-the-art compression
method with a better compression ratio and the same theoretical compression and
decompression time complexity as the previous state-of-the-art was designed. It
combines a novel encoding of the signal data's zig-zag deltas with
domain-specific stall encoding and predictive range coding. An alternative
static huffman method is proposed which also has a higher compression ratio than
the state-of-the-art but not as much as the range coding method. The first known
software benchmark of nanopore signal compression methods was written and
provided for public access.

\section{Thesis Outline}
