\chapter{Evaluation} \label{chap:evaluation}

\section{Data} \label{sec:data}

The data consists of many sequences of unsigned integers known as \textit{reads}. Each integer represents the quantised ionic current recorded at a certain time step as a single-stranded DNA or RNA molecule is driven through a nanoscale protein pore (or \textit{nanopore}) \cite{Wang2021}. Disturbances in the ionic current caused by the molecule's biological structure can be used to determine its nucleic acid sequence.

Let each read be represented by
\[ x := (x_i\mid x_i \in \mathbb{Z} \cap [0, 2^{16})) \]
where $i\in \mathbb{Z}\cap [0, n)$. Computationally, $x$ is an unsigned 16-bit integer array with $n$ elements. However, in practice the full range of $2^{16}$ integers is never met. Instead, each integer can be represented more space-efficiently by using $b<16$ bits where \[b(x)=\lceil\log_2(\max(x))\rceil.\] In this case, the compression ratio would be approximately $16/b$:
\begin{align*}
	\text{Compression Ratio} &= \frac{\text{Uncompressed Bytes}}{\text{Compressed Bytes}}\\
	&=\frac{\frac{16}{8}n}{\lceil\frac{b}{8}n\rceil}\\
	&\stackrel{n\to\infty}{\longrightarrow}\frac{16}{b}.
\end{align*}
A better strategy would be to initially transform the data such that its range is decreased.
%Let \[T=\{t\mid t:X\to Y\}\] be the set of bijections from $X$, the set of reads, to $Y := \{(y_i\mid y_i\in \mathbb{N}_0)\}$. One such transformation is defined by
Let \[T=\{t\mid t:X\to X\}\] be the set of functions from $X$, the set of reads, to itself. One such transformation subtracts the minimum of $x$ from each integer and is defined by \[ submin(x) := (x_i-\min(x)). \] Each transformed integer can then be represented using fewer than or the same number of bits:
%A better strategy would be to first transform $x$ by subtracting the minimum of $x$ from each integer. That is, $x\mapsto(x_i-\min(x))$. Then,
\begin{align*}
	b(submin(x))&=\lceil\log_2(\max(submin(x)))\rceil\\
	&=\lceil\log_2(\max(x-\min(x)))\rceil\\
	&=\lceil\log_2(\max(x)-\min(x))\rceil\\
	&\le b(x)
\end{align*}
since $\log$ is an increasing function.
%However, in practice each integer lies within $[2^7,2^{11}]$. This means each integer can be represented using 11 bits rather than 16. If each integer is stored using $b$ bits, the compression ratio would be approximately $16/b$ as follows:
%Hence, using 11 bits per integer results in a compression ratio of approximately $1.\overline{45}$. Note that the optimal $b$ is given by
Another transformation takes the differences between successive signals and is defined by
\[ \delta(x):=(x_{i+1}-x_i\mid 0\le i\le n-2).\]
% TODO analyse delta

\subsection{Subsequence Searching}
Some of the previously discussed compression methods, such as bit packing and FOR, depend on global statistics of the data such as the minimum and maximum.
For nanopore signal data, these statistics are easily dominated by outliers in the data.
One naive solution is to separately compress adjacent subsequences of equal length.
This approach has previously been successful in the literature with methods such as SIMD-BP128 and fast patched frame-of-reference (FastPFOR) performing compression in blocks of 128 integers \cite{lemire-simd}.
However, we can exploit the characteristics of nanopore signal data to find better subsequences.
For example, each read typically consists of an adapter stall which is long and has a small range.

The next idea is to divide $x$ into subsequences with small range and compress each subsequence separately. Let $OPT(i, j)$ be the minimum number of bytes after compressing $(x_i,\dots, x_j)$. Then,
\[ OPT(i, j) = \min(bytes(i, j),\min_{i\le k<j}(OPT(i, k) + OPT(k+1, j))) \]
