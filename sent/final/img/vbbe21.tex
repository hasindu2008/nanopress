%\begin{figure}
%\centering
	\begin{tikzpicture}[node distance=0cm,start chain=1 going right] \footnotesize
  \tikzstyle{mytape}=[draw,minimum height=1.7cm]
	\node(A1)  [on chain=1,mytape,fill=blue!20] {$\underset{\text{exceptions}}{\underbrace{\overbracket{\text{ }e\text{ }}^{\text{2 bytes}}}_{\text{number of}}}$};
	\node(B1)  [on chain=1,mytape,fill=yellow!20] {$\underset{\text{position}}{\underbrace{\overbracket{b_p}^{\text{1 byte}}}_{\text{bits per exception}}}$};
	\node(B2)  [on chain=1,mytape,fill=yellow!20] {$\underbrace{\overbracket{p1,\delta(p_1,p_2,\dots,p_e)-1}^{b_pe\text{ bits}}}_{\text{exception positions}}$};
	\node [on chain=1,mytape,fill=gray!35] {$\underbrace{\overbracket{\text{ }\dots \text{ }}^{<1\text{ byte}}}_{\text{padding}}$};
	\node(C1)  [on chain=1,mytape,fill=yellow!35] {$\underset{\text{byte exception}}{\underbrace{\overbracket{b_x}^{\text{1 byte}}}_{\text{bits per two}}}$};
	\node(C2)  [on chain=1,mytape,fill=yellow!35] {$\underbrace{\overbracket{(x_{p_1},x_{p_2},\dots,x_{p_e})-256}^{b_xe\text{ bits}}}_{\text{two byte exceptions}}$};
	\node [on chain=1,mytape,fill=gray!35] {$\underbrace{\overbracket{\text{ }\dots \text{ }}^{<1\text{ byte}}}_{\text{padding}}$};
	\node(D)  [on chain=1,mytape,fill=green!35] {$\underbrace{\overbracket{x_{q_1},x_{q_2},\dots,x_{q_{n-e}}}^{n-e\text{ bytes}}}_{\text{one byte data}}$};
\end{tikzpicture}
%	\caption[The vbbe21 encoding.]{\label{fig:vbbe21}The vbbe21 encoding takes $n$ unsigned 16-bit
%	integers and finds those which cannot fit into one byte. These are
%	encoded by storing the number of exceptions using two bytes, bit packing
%	the deltas of the exceptions' positions and bit packing the exceptions
%	themselves. Bit packing is performed using one byte for the number of
%	bits and padding is used between the exception's positions, data and the
%	one byte data. This is easier to implement than compact vbbe21.}
%\end{figure}

